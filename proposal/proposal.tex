\documentclass {article}
\usepackage{fullpage}

\begin{document}

~\vfill
\begin{center}
\Large

A5 Revised Project Proposal

Title: Shark Hunt - An OpenGL Interactive Experience

Name: Phong Pham

Student ID: 20610310

User ID: pt2pham
\end{center}
\vfill ~\vfill~
\newpage
\noindent{\Large \bf Final Project:}
\begin{description}
\item[Purpose]:\\
     To implement various graphics objectives that end up creating an OpenGL game that
     showcases my understanding and work in graphics this term.

\item[Statement]:\\
     This project will be an OpenGL game. The main idea of the game is 
     there will be a big body of water encompassed by a \textbf{Skybox}. Underneath 
     the water will be a \textbf{Model of a Shark} which will be \textbf{Animated to be
     swimming around} trying to kill you. As the player, you have to shoot it 
     down with enough exploding cannonballs to beat the game before you die. 
     
     In terms of how this project will map to my objectives, the shark
     approaching the player and the cannonball hit will be 
     calculated by \textbf{3D Collision Detection}. The models (such as the cannonball) will have 
     \textbf{Texture Mapping} to colour them more realistically. \textbf{Bump Mapping} 
     will be used to create ripple effects whenever a cannonball misses 
     and shoots into the water. To simulate the cannonball explosions
     upon colliding with the shark, a \textbf{Particle System} will be used. Finally, 
     the water will be \textbf{reflective and refractive} to look closer to a real body of 
     water. This should include reflection of the skybox as well as any projectiles flying over the water, 
     and projectiles / the shark that are underwater. 

     I think this is interesting because it encompasses many of the concepts talked 
     about this term and applying it to making an interactive game, which turns 
     the theoretical knowledge accumulated through the course into something that's 
     playable.
     
     I think one of the most difficult things to do will be how light works with the large body 
     of water where the game will take place. By not using ray tracing, a lot of the 
     reflection and refraction literature I've read up on does not apply. Figuring out 
     how to make the water look more realistic will be the hardest part to this, as well as 
     doing it in an optimized way that doesn't slow down my game beyond playability.

\item[Technical Outline]:\\
     I will be looking to research papers to learn how to implement many of these 
     objectives, along with the course notes. Many of these are straight forward based on 
     course learnings (like Texture and Bump mapping). 

     For 3D Collision, I will be using the axis-aligned bounding boxes concept and look for 
     when the bounding boxes of two models intersect each other. Namely, this will be around 
     the shark, the player and the weapons.

     For water to look accurate, it must support reflection and refraction. One real property 
     of how water interacts with light is that depending on the viewing angle, water will 
     appear either more reflective or more refractive. To simulate this, I will use Fresnel equations. 
     As extra objectives I will try to get the Fresnel effect working, as well as depth. In water, underwater 
     light absorption creates the phenomena that scatters some light back out of the water, and absorbs the rest. 
     This changes with the depth of water. 

\item[Bibliography]:

Many details of extensions have been looked at in the course notes.

Cai, P. \& Cai, Yiyu \& Chandrasekaran, I. \& Zheng, J.. (2013). Collision detection 
     using axis aligned bounding boxes. \textit{Simulation, Serious Games and Their Applications}. 1-14.

Iglesias, A. (2004). Computer graphics for water modeling
     and rendering: a survey. \textit{Future Generation Computer Systems}
     20, 8, 1355–1374.

Jiménez, P., Thomas, F., \& Torras, C. (2001). 3D collision detection: a survey. \textit{Computers \& Graphics}, 
     25(2), 269–285. doi: 10.1016/s0097-8493(00)00130-8

Fleck, B. (n.d.). Real-Time Rendering of Water in Computer Graphics. Seminar (Mit Bachelorarbeit) WS, 4h, 
     186(162). Retrieved from \\ 
     https://pdfs.semanticscholar.org/a101/3c7818efdd421775dc6e754ae6eb429641f2.pdf

Reeves, W. T. (1983). Particle systems---a technique for modeling a class of fuzzy 
     objects. \textit{Proceedings of the 10th Annual Conference on Computer Graphics and 
     Interactive Techniques - SIGGRAPH 83}, 17(3), 359–375. doi: 10.1145/800059.801167

Xu, X., \& Zou, K. (2012). Simulation of Realistic Water on 3D Game Scene. 
     \textit{Procedia Engineering}, 29, 1819–1823. doi: 10.1016/j.proeng.2012.01.219

\end{description}
\newpage


\noindent{\Large\bf Objectives:}

{\hfill{\bf Full UserID: pt2pham}\hfill{\bf Student ID: 20610310}\hfill}

\begin{enumerate}
     \item[\_\_\_ 1:]  Implement a skybox that positions itself around the camera and moves with it.

     \item[\_\_\_ 2:]  Create a model of a shark that will be used as the main enemy in the game.

     \item[\_\_\_ 3:]  Texture Mapping

     \item[\_\_\_ 4:]  Bump Mapping

     \item[\_\_\_ 5:]  Particle System

     \item[\_\_\_ 6:]  3D Collision Detection

     \item[\_\_\_ 7:]  Animate the shark swimming underneath water.

     \item[\_\_\_ 8:]  Implement the reflection of light on top of the water.

     \item[\_\_\_ 9:]  Implement refraction of light for underwater objects.

     \item[\_\_\_ 10:]  Final Modelled Scene of the Game
\end{enumerate}
\end{document}